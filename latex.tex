\documentclass[12pt]{article}
\usepackage{amsmath}
\usepackage{graphicx}
\usepackage{geometry}
\usepackage{float}
\usepackage{booktabs}
\usepackage{tikz}
\usepackage{hyperref}
\geometry{a4paper, margin=1in}

\title{Airfoil Redesign Project}
\author{Thomas Xiao}
\date{\today}

\begin{document}

\maketitle

\section*{1. Introduction}

The goal of this project is to redesign an airfoil to generate more lift without adding additional material. The lift generated by a wing depends on several factors, including the shape of the wing, air density, and velocity of the aircraft. Out of these factors, only the wing shape can be controlled during the design process. By modifying the airfoil geometry, we aim to increase the lift while adhering to material constraints and ensuring the design operates within safe aerodynamic limits.

\section*{2. Current Wing Design}

\subsection*{2.1 Description of the Current Design}

The current airfoil design consists of two intersecting tapered parabolic cylinders described by the following equations:
\begin{itemize}
    \item Upper surface: \( z_{\text{top}}(x, y) = -y  \left( \frac{y}{c(x)} - 1 \right) \)
    \item Lower surface: \( z_{\text{bot}}(x, y) = \frac{y}{2} \left( \frac{y}{c(x)} - 1 \right) \)
    \item Chord length: \( c(x) = c_{\text{root}} \left( 1 - \frac{x}{L} \right) + c_{\text{tip}}  \left(\frac{x}{L}\right) \)
\end{itemize}

The parameters for the design are as follows:
\begin{itemize}
    \item Wing length: \( L = 15 \text{ ft} \)
    \item Root chord: \( c_{\text{root}} = 5 \text{ ft} \)
    \item Tip chord: \( c_{\text{tip}} = 2 \text{ ft} \)
\end{itemize}

\subsection*{2.2 Graphical Representation}

The cross-sectional view of the airfoil and various views of the wing design are shown below. All graphs are restricted to $0 \leq x \leq 15$, $y \leq 5 - \frac{x}{5}$, and $-y \leq \frac{y}{2}$ as per their intersection lines.
\begin{figure}[H]
    \centering
    \begin{tikzpicture}
        \node[anchor=south west, inner sep=0] (image) at (0,0) {\includegraphics[width=0.8\textwidth]{cross_section.png}};
        \begin{scope}[x={(image.south east)},y={(image.north west)}]
            % Add labels for the axes
            \node[anchor=west, text=green] at (0.8, 0.25) {\Large Width (ft)};
            \node[anchor=north, text=blue] at (0.4, 0.85) {\Large Height (ft)};
        \end{scope}
    \end{tikzpicture}
    \caption{Cross-sectional view of the current wing design.}
\end{figure}


\begin{figure}[H]
    \centering
    \begin{tikzpicture}
        \node[anchor=south west, inner sep=0] (image) at (0,0) {\includegraphics[width=0.8\textwidth]{left_view.png}};
        \begin{scope}[x={(image.south east)},y={(image.north west)}]
            % Add labels for the axes
            \node[anchor=west, text=red] at (0.8, 0.05) {\Large Length (ft)};
            \node[anchor=west, text=green] at (0.7, 0.77) {\Large Width (ft)};
            \node[anchor=west, text=blue] at (0.15,0.85) {\Large Height (ft)};
        \end{scope}
    \end{tikzpicture}
    \caption{Left view of the current wing design.}
\end{figure}

\begin{figure}[H]
    \centering
    \begin{tikzpicture}
        \node[anchor=south west, inner sep=0] (image) at (0,0) {\includegraphics[width=0.8\textwidth]{top_view.png}};
        \begin{scope}[x={(image.south east)},y={(image.north west)}]
            % Add labels for the axes
            \node[anchor=west, text=red] at (0.75, 0.23) {\Large Length (ft)};
            \node[anchor=west, text=green] at (0.2, 0.85) {\Large Width (ft)};
        \end{scope}
    \end{tikzpicture}
    \caption{Top view of the current wing design.}
\end{figure}

\begin{figure}[H]
    \centering
    \begin{tikzpicture}
        \node[anchor=south west, inner sep=0] (image) at (0,0) {\includegraphics[width=0.8\textwidth]{right_view.png}};
        \begin{scope}[x={(image.south east)},y={(image.north west)}]
            % Add labels for the axes
            \node[anchor=west, text=green] at (0.75, 0.35) {\Large Width (ft)};
            \node[anchor=west, text=red] at (0.1, 0.45) {\Large Length (ft)};
            \node[anchor=east, text=blue] at (0.6,0.85) {\Large Height (ft)};
        \end{scope}
    \end{tikzpicture}
    \caption{Right view of the current wing design.}
\end{figure}

\begin{figure}[H]
    \centering
    \includegraphics[scale=0.5]{prompt.png}
    \caption{Specific GeoGebra 3D input to produce the current design}
    \label{fig:4}
\end{figure}




\section*{3. Volume Calculation of the Current Design}
The volume of the current wing design can be calculated by integrating over the entire three-dimensional region bounded by the upper and lower surfaces. Using a triple integral, the volume is expressed as:
\[
V = \int_{0}^L \int_{0}^{c(x)} \int_{z_{\text{bot}}(x, y)}^{z_{\text{top}}(x, y)} 1 \, \mathrm{d}z \, \mathrm{d}y \, \mathrm{d}x.
\]

Here, 
\begin{itemize}
    \item \( L = 15 \text{ ft} \) 
    \item \( c_{\text{root}} = 5 \text{ ft}\)
    \item \( c_{\text{tip}} = 2 \text{ ft}\)
    \item \( c(x) = c_{\text{root}} \left( 1 - \frac{x}{L} \right) + c_{\text{tip}} \left( \frac{x}{L} \right) \) is the chord length as a function of \( x \).
    \item \( z_{\text{top}}(x, y) = -y \left( \frac{y}{c(x)} - 1 \right) \) is the upper surface.
    \item \( z_{\text{bot}}(x, y) = \frac{y}{2} \left( \frac{y}{c(x)} - 1 \right) \) is the lower surface.
\end{itemize}



The volume of the current wing design is calculated as:
\[
V = \int_{0}^{15} \int_{0}^{5 - \frac{x}{5}} \int_{\frac{y}{2} \left( \frac{5y}{25 - x} - 1 \right)}^{-y \left( \frac{5y}{25 - x} - 1 \right)} 1 \, \mathrm{d}z \, \mathrm{d}y \, \mathrm{d}x.
\]

\[
V = \int_{0}^{15} \int_{0}^{5 - \frac{x}{5}} \left[ -y \left( \frac{5y}{25 - x} - 1 \right) - \frac{y}{2} \left( \frac{5y}{25 - x} - 1 \right) \right] \, \mathrm{d}y \, \mathrm{d}x.
\]

\[
V = \int_{0}^{15} \int_{0}^{5 - \frac{x}{5}} \left[ -\frac{3y}{2} \left( \frac{5y}{25 - x} - 1 \right) \right] \, \mathrm{d}y \, \mathrm{d}x.
\]

\[
V = \int_{0}^{15} \int_{0}^{5 - \frac{x}{5}} \left[ -\frac{15y^2}{2(25 - x)} + \frac{3y}{2} \right] \, \mathrm{d}y \, \mathrm{d}x.
\]

\[
V = \int_{0}^{15} \left[ \int_{0}^{5 - \frac{x}{5}} -\frac{15y^2}{2(25 - x)} \, \mathrm{d}y + \int_{0}^{5 - \frac{x}{5}} \frac{3y}{2} \, \mathrm{d}y \right] \, \mathrm{d}x.
\]

\[
V = \int_{0}^{15} \left[ -\frac{15}{2(25 - x)} \cdot \frac{\left( 5 - \frac{x}{5} \right)^3}{3} + \frac{3}{2} \cdot \frac{\left( 5 - \frac{x}{5} \right)^2}{2} \right] \, \mathrm{d}x.
\]

\[
V = \int_{0}^{15} \left[ -\frac{5 \left( 5 - \frac{x}{5} \right)^3}{2(25 - x)} + \frac{3 \left( 5 - \frac{x}{5} \right)^2}{4} \right] \, \mathrm{d}x.
\]

\[
V = \int_{0}^{15} \left[ -\frac{5}{2(25 - x)} \cdot \left( 125 - \frac{75x}{5} + \frac{15x^2}{25} - \frac{x^3}{125} \right) + \frac{3}{4} \cdot \left( 25 - \frac{10x}{5} + \frac{x^2}{25} \right) \right] \, \mathrm{d}x.
\]

\[
V = \int_{0}^{15} \left[ -\frac{5}{2(25 - x)} \cdot \left( 125 - 15x + \frac{3x^2}{5} - \frac{x^3}{125} \right) + \frac{3}{4} \cdot \left( 25 - 2x + \frac{x^2}{25} \right) \right] \, \mathrm{d}x.
\]


\[
V = \int_{0}^{15} \Bigg[ 
-\frac{625}{2(25 - x)} 
+ \frac{75x}{2(25 - x)} 
- \frac{3x^2}{2(25 - x)} 
+ \frac{x^3}{50(25 - x)} 
\Bigg] \, \mathrm{d}x
\]
\[
+ \int_{0}^{15} \Bigg[ 
\frac{75}{4} 
- \frac{3x}{2} 
+ \frac{3x^2}{100} 
\Bigg] \, \mathrm{d}x.
\]
\small{
\[  \int_{0}^{15}
-\frac{625}{2(25 - x)} \, \mathrm{d}x = -\frac{625}{2} \int_{0}^{15} \frac{1}{25 - x} \, \mathrm{d}x = \frac{625}{2} \ln(25 - x) \Bigg |^{15}_{0} = \frac{625}{2}\Bigg(\ln 10 - \ln25\Bigg)  \]}

\[ \int_{0}^{15} \frac{75x}{2(25-x)} \, \mathrm{d}x = \frac{75}{2} \int_{0}^{15} \frac{x}{25 - x} \, \mathrm{d}x \]

Let $u = 25 - x$. Then, $\mathrm{d}u = -\mathrm{d}x.$
\small{
\[-\frac{75}{2} \int_{25}^{10} \frac{25 - u}{u} \, \mathrm{d}u = -\frac{75}{2}\Bigg(25\ln(u) - u\Bigg) \Bigg |_{25}^{10} = -\frac{75}{2}\Bigg( 15 + 25 \ln 10 - 25 \ln 25\Bigg)\]}


\[
\int_{0}^{15} -\frac{3x^2}{2(25 - x)} \,dx = -\frac{3}{2} \int_{0}^{15} \frac{x^2}{25 - x} \,dx
\]

Let \( u = 25 - x \). Then, \( \mathrm{d}u = -\mathrm{d}x \).

\small{
\[
\frac{3}{2} \int_{25}^{10} \frac{(25 - u)^2}{u} \, \mathrm{d}u = \frac{3}{2} \int_{25}^{10} \frac{625 - 50u + u^2}{u} \,\mathrm{d}u
\]
\[
\frac{3}{2} \int_{25}^{10} \frac{(25 - u)^2}{u} \, \mathrm{d}u  = \frac{3}{2} \Bigg[ \int_{25}^{10} \frac{625}{u} \,\mathrm{d}u - 50 \int_{25}^{10} \,\mathrm{d}u + \int_{25}^{10} u \,\mathrm{d}u \Bigg]
\]
\[
\frac{3}{2} \int_{25}^{10} \frac{(25 - u)^2}{u} \, \mathrm{d}u = \frac{3}{2} \Bigg[ 625 (\ln 10 - \ln 25) + 50(15) + \frac{10^2}{2} - \frac{25^2}{2} \Bigg]
\]

\[
\frac{3}{2} \int_{25}^{10} \frac{(25 - u)^2}{u} \, \mathrm{d}u  = \frac{3}{2} \Bigg[ 625 \ln 10 - 625 \ln 25 +487.5\Bigg]
\]

}

\[
\int_{0}^{15} \frac{x^3}{50(25 - x)} \,dx = \frac{1}{50} \int_{0}^{15} \frac{x^3}{25 - x} \,dx
\]

Let \( u = 25 - x \). Then, \( \mathrm{d}u = -\mathrm{d}x \).

\small{
\[
-\frac{1}{50} \int_{25}^{10} \frac{(25 - u)^3}{u} \, \mathrm{d}u = \frac{1}{50} \int_{10}^{25} \frac{15625 - 1875u + 75u^2 - u^3}{u} \,\mathrm{d}u
\]
\[
-\frac{1}{50} \int_{25}^{10} \frac{(25 - u)^3}{u} \, \mathrm{d}u = \frac{1}{50} \Bigg[ \int_{10}^{25} \frac{15625}{u} \,\mathrm{d}u - 1875 \int_{10}^{25} \,\mathrm{d}u + 75 \int_{10}^{25} u \,\mathrm{d}u - \int_{10}^{25} u^2 \,\mathrm{d}u \Bigg]
\]
\[
-\frac{1}{50} \int_{25}^{10} \frac{(25 - u)^3}{u} \, \mathrm{d}u = \frac{1}{50} \Bigg[ 15625 (\ln 25 - \ln 10) - 1875(15) + 75 \left( \frac{25^2}{2} - \frac{10^2}{2} \right) - \left( \frac{25^3}{3} - \frac{10^3}{3} \right) \Bigg]
\]

\[
-\frac{1}{50} \int_{25}^{10} \frac{(25 - u)^3}{u} \, \mathrm{d}u = \frac{1}{50} \Bigg[ 15625(\ln 25 - \ln 10) - 13312.5\Bigg]
\]
}

\[
\int_{0}^{15} \Bigg[ 
\frac{75}{4} 
- \frac{3x}{2} 
+ \frac{3x^2}{100} 
\Bigg] \,dx = \frac{75}{4}(15) - \frac{3}{2}\left(\frac{15^2}{2}\right) + \frac{3}{100}\left(\frac{15^3}{3}\right)\]
\small{
\[V = \frac{625}{2}\Bigg(\ln(10) - \ln(25)\Bigg) -\frac{75}{2}\Bigg( 15 + 25 \ln(10) - 25 \ln(15)\Bigg)\] 
\[ + \frac{3}{2} \Bigg[ 625 \ln 10 - 625 \ln 25 +487.5\Bigg] + \frac{1}{50}\Bigg[ 15625(\ln 25 - \ln 10) - 13312.5\Bigg]\]  \[+\frac{75}{4}(15) - \frac{3}{2}\left(\frac{15^2}{2}\right) + \frac{3}{100}\left(\frac{15^3}{3}\right)\] 
}
\[ V = \frac{195}{4} = 48.75 \text{ ft}^3\]











\section*{4. Surface Area Calculation of the Current Design}
The surface area of the current wing design can be calculated by finding the surface area of each surface:
\[
S = \int \int \left\| \frac{\partial \mathbf{r}}{\partial x} \times \frac{\partial \mathbf{r}}{\partial y} \right\| \, \mathrm{d}x \, \mathrm{d}y,
\]
where \( \frac{\partial \mathbf{r}}{\partial x} \) and \( \frac{\partial \mathbf{r}}{\partial y} \) are the partial derivatives of the position vector \( \mathbf{r}(x, y) \) with respect to \(x\) and \(y\) that represent the parameterized surface.

\subsection*{4.1 Upper Surface Area}

For the upper surface, the parametric representation is:
\[
\mathbf{r}_{\text{top}}(x, y) = \langle x, y, z_{\text{top}}(x, y) \rangle,
\]
where \( z_{\text{top}}(x, y) = -y \left( \frac{y}{c(x)} - 1 \right) \) and \( c(x) = c_{\text{root}} \left( 1 - \frac{x}{L} \right) + c_{\text{tip}} \frac{x}{L} \). The partial derivatives are:
\[
\frac{\partial \mathbf{r}_{\text{top}}}{\partial x} = \langle 1, 0, \frac{\partial z_{\text{top}}}{\partial x} \rangle, \quad
\frac{\partial \mathbf{r}_{\text{top}}}{\partial y} = \langle 0, 1, \frac{\partial z_{\text{top}}}{\partial y} \rangle,
\]
where:
\[
\frac{\partial z_{\text{top}}}{\partial x} = \frac{y^2}{c(x)^2} \frac{\partial c(x)}{\partial x} = \frac{y^2}{(5 - \frac{x}{5})^2} \left(-\frac{1}{5}\right) = \frac{-y^2}{5(5 - \frac{x}{5})^2}\] 


\[
\quad \frac{\partial z_{\text{top}}}{\partial y} = -\frac{2y}{c(x)} + 1 = -\frac{2y}{5 - \frac{x}{5}} = -\frac{10y}{25 - x}
\]

The cross product is:
\[
\frac{\partial \mathbf{r}_{\text{top}}}{\partial x} \times \frac{\partial \mathbf{r}_{\text{top}}}{\partial y} =
\begin{vmatrix}
\mathbf{i} & \mathbf{j} & \mathbf{k} \\
1 & 0 & \frac{\partial z_{\text{top}}}{\partial x} \\
0 & 1 & \frac{\partial z_{\text{top}}}{\partial y}
\end{vmatrix} = \langle -\frac{\partial z_{\text{top}}}{\partial x}, -\frac{\partial z_{\text{top}}}{\partial y}, 1 \rangle.
\]

The magnitude of the cross product is:
\[
\left\| \frac{\partial \mathbf{r}_{\text{top}}}{\partial x} \times \frac{\partial \mathbf{r}_{\text{top}}}{\partial y} \right\| =
\sqrt{1 + \left( \frac{\partial z_{\text{top}}}{\partial x} \right)^2 + \left( \frac{\partial z_{\text{top}}}{\partial y} \right)^2}.
\]

The surface area of the upper surface is:
\[
S_{\text{top}} = \int_{0}^L \int_{0}^{c(x)} \sqrt{1 + \left( \frac{\partial z_{\text{top}}}{\partial x} \right)^2 + \left( \frac{\partial z_{\text{top}}}{\partial y} \right)^2} \, \mathrm{d}y \, \mathrm{d}x.
\]

\[
S_{\text{top}} = \int_{0}^{15} \int_{0}^{5 - \frac{x}{5}} \sqrt{1 + \left( \frac{y^2}{5(5 - \frac{x}{5})^2} \right)^2 + \left( \frac{10y}{25 - x} \right)^2} \, \mathrm{d}y \, \mathrm{d}x.
\]

\[
S_{\text{top}} = \int_{0}^{15} \int_{0}^{5 - \frac{x}{5}} \sqrt{1 + \left( \frac{5y^2}{25(5 - \frac{x}{5})^2} \right)^2 + \left( \frac{10y}{25 - x} \right)^2} \, \mathrm{d}y \, \mathrm{d}x.
\]

\[
S_{\text{top}} = \int_{0}^{15} \int_{0}^{5 - \frac{x}{5}} \sqrt{1 + \frac{25y^4}{(25 - x)^4} +  \frac{100y^2}{(25 - x)^2}} \, \mathrm{d}y \, \mathrm{d}x.
\]

\[S_{\text{top}} \approx 77.75 \text{ ft}^2\]

\subsection*{4.2 Lower Surface Area}

For the lower surface, the parametric representation is:
\[
\mathbf{r}_{\text{bot}}(x, y) = \langle x, y, z_{\text{bot}}(x, y) \rangle,
\]
where \( z_{\text{bot}}(x, y) = \frac{y}{2} \left( \frac{y}{c(x)} - 1 \right) \).

The partial derivatives are:
\[
\frac{\partial \mathbf{r}_{\text{bot}}}{\partial x} = \langle 1, 0, \frac{\partial z_{\text{bot}}}{\partial x} \rangle, \quad
\frac{\partial \mathbf{r}_{\text{bot}}}{\partial y} = \langle 0, 1, \frac{\partial z_{\text{bot}}}{\partial y} \rangle,
\]
where:
\[
\frac{\partial z_{\text{bot}}}{\partial x} = -\frac{y^2}{2c(x)^2} \frac{\partial c(x)}{\partial x} = -\frac{y^2}{2(5 - \frac{x}{5})^2} \left(-\frac{1}{5}\right) = \frac{y^2}{10(5 - \frac{x}{5})^2} = \frac{10y^2}{(50-2x)^2}
\]

\[\frac{\partial z_{\text{bot}}}{\partial y} = \frac{y}{c(x)} - \frac{1}{2} = \frac{y}{5 - \frac{x}{5}} - \frac{1}{2} = \frac{5y - \frac{25 - x}{2}}{25 - x} = \frac{x + 10y - 25}{50 - 2x} \]


Similar to the upper surface area, the surface area of the lower surface is:
\[
S_{\text{bot}} = \int_{0}^L \int_{0}^{c(x)} \sqrt{1 + \left( \frac{\partial z_{\text{bot}}}{\partial x} \right)^2 + \left( \frac{\partial z_{\text{bot}}}{\partial y} \right)^2} \, \mathrm{d}y \, \mathrm{d}x.
\]

\[
S_{\text{bot}} = \int_{0}^{15} \int_{0}^{5 - \frac{x}{5}} \sqrt{1 + \left( \frac{10y^2}{(50-2x)^2} \right)^2 + \left( \frac{x + 10y - 25}{50 - 2x} \right)^2} \, \mathrm{d}y \, \mathrm{d}x.
\]

\[
S_{\text{bot}} \approx 54.7 \text{ ft}^2 \]
\subsection*{4.3 Total Surface Area}
The expressions for $S_{\text{top}}$ and $S_{\text{bot}}$ could not be explicitly integrated, and Wolfram Alpha was used to make numerical approximations. Either way, the total surface area of the wing is the sum of the surface areas of the upper and lower surfaces:
\[
S_{\text{total}} = S_{\text{top}} + S_{\text{bot}} \approx 77.75 \, \text{ft}^2 + 54.7 \, \text{ft}^2 = 132.45 \text{ ft}^2
\]

\section*{5. The New Design}

\subsection*{5.1 Design Process}

The new wing design utilizes a modified version of the NACA 0015 airfoil \cite{airfoil_intro}, scaled and adjusted to ensure that the total surface area of the wing is less than 132 square feet, satisfying the material constraints. The NACA 0015 airfoil is known for its efficient lift-to-drag ratio, making it a suitable candidate for improving lift performance under the given conditions. Adjustments to the airfoil geometry have been made to reduce the surface area while retaining the favorable aerodynamic characteristics of the original NACA design. Here are some key considerations in the design process:
\begin{itemize}
    \item \textbf{Efficient Airfoil Shape:} The NACA airfoil provides a well-balanced lift-to-drag ratio, making it an ideal choice for optimizing aerodynamic performance.
    \item \textbf{Scaling and Adjustment:} The airfoil's dimensions were scaled down and modified slightly to ensure the surface area remains within the specified constraints.
    \item \textbf{Maintaining Structural Compatibility:} The chord length and wing span were adjusted to match the existing structural design parameters of the aircraft.
\end{itemize}

\subsection*{5.2 New Design Equations}

The equations for the new wing design, based on the modified NACA 2412 airfoil, are as follows:
\begin{itemize}
    \item \textbf{Upper Surface:}
    \[
z_{\text{top,new}}(x, y) = 
\]
{\small
\[
c(x)\Bigg( 
0.3 \sqrt{\frac{y}{c(x)}} \\
- 0.13 \frac{y}{c(x)} \\
- 0.35 \left(\frac{y}{c(x)}\right)^2 \\
+ 0.28 \left(\frac{y}{c(x)}\right)^3 \\
- 0.1\left(\frac{y}{c(x)}\right)^4 
\Bigg)
\]
}



    where \( c(x) = c_{\text{root}} \left( 1 - \frac{x}{L} \right) + c_{\text{tip}} \frac{x}{L} \) is the chord length at a given spanwise position \( x \).
    \item \textbf{Lower Surface:}
    \[
    z_{\text{bot,new}}(x, y) = -z_{\text{top,new}}(x, y)
    \]
    reflecting the symmetric scaling of the NACA airfoil geometry.
    
\end{itemize}

The parameters for the new design remain the same:
\begin{itemize}
    \item Wing span: \( L = 15 \, \text{ft} \),
    \item Root chord: \( c_{\text{root}} = 5 \, \text{ft} \),
    \item Tip chord: \( c_{\text{tip}} = 2 \, \text{ft} \).
\end{itemize}







\subsection*{5.3 Graphical Representation of the New Design}

The cross-sectional view of the new design and various views of the design are shown
below. Similar to before, all graphs are restricted to $0 \leq x \leq 15$ and $y \leq 5 - \frac{x}{5}$
\begin{figure}[H]
    \centering
    \begin{tikzpicture}
        \node[anchor=south west, inner sep=0] (image) at (0,0) {\includegraphics[scale=0.75]{new_cross}};
        \begin{scope}[x={(image.south east)},y={(image.north west)}]
            % Add labels for the axes
            \node[anchor=west, text=green] at (0.85, 0.33) {\Large Width (ft)};
            \node[anchor=north, text=blue] at (0.53, 0.85) {\Large Height (ft)};
        \end{scope}
    \end{tikzpicture}
    \caption{Cross-sectional view of the new design}
    \label{fig:5}
\end{figure}

\begin{figure}[H]
    \centering
    \begin{tikzpicture}
        \node[anchor=south west, inner sep=0] (image) at (0,0) {\includegraphics[scale=0.75]{new_left}};
        \begin{scope}[x={(image.south east)},y={(image.north west)}]
            % Add labels for the axes
            \node[anchor=west, text=red] at (0.4, 0.25) {\Large Length(ft)};
             \node[anchor=east, text=green] at (1.2, 0.8) {\Large Width(ft)};
            \node[anchor=north, text=blue] at (0.05, 1) {\Large Height(ft)};
        \end{scope}
    \end{tikzpicture}
    \caption{Left view of the new design}
    \label{fig:6}
\end{figure}

\begin{figure}[H]
    \centering
    \begin{tikzpicture}
        \node[anchor=south west, inner sep=0] (image) at (0,0) {\includegraphics[scale=0.75]{new_top}};
        \begin{scope}[x={(image.south east)},y={(image.north west)}]
            % Add labels for the axes
            \node[anchor=west, text=red] at (0.7, 0.3) {\Large Length(ft)};
            \node[anchor=north, text=green] at (0.2, 0.85) {\Large Width(ft)};
        \end{scope}
    \end{tikzpicture}
    \caption{Top view of the new design}
    \label{fig:7}
\end{figure}

\begin{figure}[H]
    \centering
    \begin{tikzpicture}
        \node[anchor=south west, inner sep=0] (image) at (0,0) {\includegraphics[scale=0.75]{new_right}};
        \begin{scope}[x={(image.south east)},y={(image.north west)}]
            % Add labels for the axes
            \node[anchor=west, text=red] at (0.2, 0.15) {\Large Length(ft)};
            \node[anchor=east, text=green] at (0.95, 0.4) {\Large Width(ft)};
            \node[anchor=north, text=blue] at (0.3, 0.85) {\Large Height(ft)};
        \end{scope}
    \end{tikzpicture}
    \caption{Right view of the new design}
    \label{fig:8}
\end{figure}

\begin{figure}[H]
    \centering
    \includegraphics[scale=0.5]{prompt2.png}
    \caption{Specific GeoGebra 3D input to produce the new design}
    \label{fig:4}
\end{figure}

\section*{6. Volume Calculation of the New Design}
Similar to the previous design, the volume is expressed as:
\[
V = \int_{0}^L \int_{0}^{c(x)} \int_{z_{\text{bot,new}}(x, y)}^{z_{\text{top,new}}(x, y)} 1 \, \mathrm{d}z \, \mathrm{d}y \, \mathrm{d}x.
\]

Here, 
\begin{itemize}
    \item \( L = 15 \text{ ft} \) 
    \item \( c_{\text{root}} = 5 \text{ ft}\)
    \item \( c_{\text{tip}} = 2 \text{ ft}\)
    \item \( c(x) = c_{\text{root}} \left( 1 - \frac{x}{L} \right) + c_{\text{tip}} \left( \frac{x}{L} \right) \) is the chord length as a function of \( x \).
    \item \( z_{\text{top,new}}(x, y) \) is the new upper surface.
    \item \( z_{\text{bot,new}}(x, y) \) is the new lower surface.
\end{itemize}



The volume of the new wing design is calculated as:
\[
V = \int_{0}^{15} \int_{0}^{5 - \frac{x}{5}} \int_{ z_{\text{bot,new}}(x, y)}^{ z_{\text{top,new}}(x, y)} 1 \, \mathrm{d}z \, \mathrm{d}y \, \mathrm{d}x
\]

\[
V = \int_{0}^{15} \int_{0}^{5 - \frac{x}{5}} z_{\text{top,new}}(x, y) - z_{\text{bot,new}(x, y)} \,  \, \mathrm{d}y \, \mathrm{d}x
\]

Since $z_{\text{bot,new}} = -z_{\text{top,new}}(x, y)$, 

\[
V = 2 \int_{0}^{15} \int_{0}^{5 - \frac{x}{5}} z_{\text{top,new}}(x, y)  \,  \, \mathrm{d}y \, \mathrm{d}x
\]

\[
V = 2 \int_{0}^{15} (0.00273333x^2-0.136667x + 1.70833)\mathrm{d}x
\]

\[ V = 2\Bigg[ 0.00273\left(\frac{15^3}{3}\right) - 0.1367\left(\frac{15^2}{2}\right) + 1.70833(15)\Bigg]\]


\[V \approx 26.65 \text{ ft}^3\]
\section*{7. Surface Area Calculation of the New Design}
To approximate the surface area of the wing's upper surface, the following method was used. The surface area integral is defined as:
\[
S= \int_{0}^{L} \int_{0}^{c(x)} \sqrt{1 + \left(\frac{\partial z_{\text{top,new}}}{\partial x}\right)^2 + \left(\frac{\partial z_{\text{top,new}}}{\partial y}\right)^2} \, \mathrm{d}y \, \mathrm{d}x.
\]

However, evaluating this double integral directly was computationally intensive and could not be resolved, and I was not able to resolve this integral. Therefore, a numerical approximation was employed using discrete cross-sectional slices of the wing. The spanwise direction of the wing, \(0 \leq x \leq 15\), was divided into \(n = 1000\) equally spaced intervals. For each value of \(x\), the chordwise direction, \(0 \leq y \leq c(x)\), was divided into \(m = 1000\) equally spaced intervals, where \(c(x)\) is the chord length given by:
   \[
   c(x) = c_{\text{root}} \left( 1 - \frac{x}{L} \right) + c_{\text{tip}} \frac{x}{L}.
   \]
The partial derivatives of \(z_{\text{top,new}}(x, y)\) with respect to \(x\) and \(y\) were approximated using finite difference formulas:
   \[
   \frac{\partial z_{\text{top,new}}}{\partial x} \approx \frac{z_{\text{top,new}}(x + \Delta x, y) - z_{\text{top,new}}(x, y)}{\Delta x},
   \]
   \[
   \frac{\partial z_{\text{top,new}}}{\partial y} \approx \frac{z_{\text{top,new}}(x, y + \Delta y) - z_{\text{top,new}}(x, y)}{\Delta y}.
   \]

For each pair \((x_i, y_j)\), the local area element was computed as:
   \[
   \mathrm{d}A = \sqrt{1 + \left(\frac{\partial z_{\text{top,new}}}{\partial x}\right)^2 + \left(\frac{\partial z_{\text{top,new}}}{\partial y}\right)^2} \cdot \Delta x \cdot \Delta y.
   \]
The total surface area was approximated by summing the contributions from all grid points:
   \[
   S \approx \sum_{i=0}^{n-1} \sum_{j=0}^{m-1} \mathrm{d}A.
   \]


Using this method, the approximate surface area of the upper surface was calculated by a computer program as:
\[
S \approx 54.95 \, \text{ft}^2 
\]
Therefore, the total surface area is approximately $54.95 \, \text{ft}^2 \cdot 2 = 109.9 \text{ ft}^2$

\begin{figure}[H]
    \centering
    \includegraphics[scale=0.5]{code.png}
    \caption{Program used to approximate}
\end{figure}
\begin{figure}[H]
    \centering
    \includegraphics[scale=0.5]{program.png}
    \caption{Result of the program}
\end{figure}


\section*{8. Lift Analysis of Current and New Design}



\subsection*{8.1 Expressions for \(|\mathbf{v}_{\text{top}}|^2 \text{ and } |\mathbf{v}_{\text{bot}}|^2 \) of the current design}

$|\mathbf{v}_{\text{top}}|$, or the speed of air moving around the wing, can be expressed as 

\[\frac{L_{\text{top}}}{T}\]

where $L_{\text{top}}$ denotes the arc length of the top of a cross-section of a wing, while $T$ is the time it takes for air to traverse the entire length of the wing. More specifically, 
\[L_{\text{top}} =  \int_0^{c(x)} \sqrt{1 +  \left(\frac{\partial z_\text{top}}{\partial y}\right)^2}   \, \mathrm{d}y   \]

and 

\[T = \frac{c(x)}{|\mathbf{v}_{\infty}|}\] 

as the time it takes for air to move across the wing is the same as the time it takes the plane itself to move through the air. Using previous evaluations, 
\\
\[ \frac{\partial z_{\text{top}}}{\partial y} = -\frac{10y}{25 - x} \]
\[L_{\text{top}} =  \int_{0}^{5 - \frac{x}{5}} \sqrt{1 + \left( \frac{10y}{25 - x} \right)^2} \, \mathrm{d}y \]
Let $u = \frac{10y}{25 - x}$. Then, $\mathrm{d}u = \frac{10}{25 - x} \mathrm{d}y \implies \mathrm{d}y = \frac{25 - x}{10} \mathrm{d}u$ . When $y = 0$, $u = 0$. When $y = 5 - \frac{x}{5}$, $u = \frac{10(5 - \frac{x}{5})}{25 - x} = \frac{50-2x}{25-x} = 2$.
Therefore, 

\[L_{\text{top}} = \frac{25 - x}{10} \int_0^2 \sqrt{1 + u^2} \, \mathrm{d}u \]

Since \[\int \sqrt{1 + x^2} \, \mathrm{d}x = \frac{1}{2} (\ln(x + \sqrt{1 + x^2}) + x\sqrt{1 + x^2})\]

then

\[L_{\text{top}} = \frac{25 - x}{20}\left( \ln(2 + \sqrt{5}) + 2\sqrt{5}\right) \]

\[ |\mathbf{v_{\text{top}}}|^2 = \left(\frac{L_\text{top}}{T}\right)^2 = \left( \frac{|\mathbf{v}_{\infty}|(25 - x)( \ln(2 + \sqrt{5}) + 2\sqrt{5}) }{100 - 4x}\right)^2 =  \left( \frac{|\mathbf{v}_{\infty}|( \ln(2 + \sqrt{5}) + 2\sqrt{5}) }{4}\right)^2 \]

\[|\mathbf{v_{\text{top}}}|^2 \approx 2.19 |\mathbf{v}_{\infty}|^2  \]
$|\mathbf{v}_{\text{bot}}|$ is similarly calculated:

\[ \frac{\partial z_{\text{bot}}}{\partial y} = \frac{x + 10y - 25}{50 - 2x} \]

\[ L_{\text{bot}} = \int_0^{5 - \frac{x}{5}} \sqrt{1 +\left( \frac{x + 10y - 25}{50 - 2x}\right)^2} \, \mathrm{d}y\]

Let $u = \frac{10y + x -25}{50 - 2x}$. Then, $\mathrm{d}u = \frac{10}{50 - 2x}\mathrm{d}y \implies \mathrm{d}y = \frac{50 - 2x}{10} \mathrm{d}u$. When $y = 0$, $u = \frac{x - 25}{50 - 2x} = -\frac{1}{2}$. When $y = 5 - \frac{x}{5}$, $u = \frac{10(5 - \frac{x}{5}) + x - 25}{50 - 2x} = \frac{50 - 2x + x - 25}{50 - 2x} = 1 - \frac{1}{2} = \frac{1}{2} $

\[ L_{\text{bot}} = \frac{50 - 2x}{10}\int_{-\frac{1}{2}}^{\frac{1}{2}} \sqrt{1 + u^2} \, \mathrm{d}u\]

\[ L_{\text{bot}} = \frac{50 - 2x}{20} \left(\ln\left(\frac{1}{2} + \sqrt{\frac{5}{4}}\right) + \frac{1}{2}\sqrt{\frac{5}{4}} - \ln\left(-\frac{1}{2} + \sqrt{\frac{5}{4}}\right) + \frac{1}{2}\sqrt{\frac{5}{4}}\right) \]

\[ L_{\text{bot}} \approx 2.08\left(\frac{50 - 2x}{20}\right) = \frac{104 - 4.16x}{20} = \frac{26 - 1.04x}{5} \]

\[ |\mathbf{v_{\text{bot}}}|^2 = \left(\frac{L_\text{bot}}{T}\right)^2 \approx \left(\frac{(\mathbf{v}_{\infty})(26 - 1.04x)}{25 - x}\right)^2 = 1.0816(\mathbf{v}_{\infty})^2 \]

\begin{figure}[H]
    \centering
    \includegraphics[scale=0.4]{velocity.png}
    \caption{Velocity vectors drawn and labeled on a cross-section of current wing}
\end{figure}




\subsection*{8.2 Expressions for \(|\mathbf{v}_{\text{top}}|^2 \text{ and } |\mathbf{v}_{\text{bot}}|^2 \) of the new design}

To differentiate, let $|\mathbf{v}_{\text{top,new}}|$ be the speed of air moving around the top part of the wing. Since the wing is symmetrical, $|\mathbf{v}_{\text{bot,new}}| = |\mathbf{v}_{\text{top,new}}|$. Similar to before,

\[|\mathbf{v}_{\text{top,new}}| = \frac{L_{\text{top,new}}}{T}\]
\[ L_{\text{top,new}} = \int_0^{c(x)} \sqrt{1 + \left( \frac{\partial z_{\text{top,new}}}{\partial y}\right)^2} \, \mathrm{d}y\]



\[\quad \frac{\partial z_{\text{top,new}}}{\partial y} = \frac{0.3\sqrt{c(x)}}{2\sqrt{y}} - 0.13 - \frac{0.7y}{c(x)} + \frac{0.84y^2}{c(x)^2} - \frac{0.4y^3}{c(x)^3} \]

% \[ = \left(5 - \frac{x}{5}\right) \left( -\frac{250y^3}{(25-x)^4} + \frac{105y^2}{(25-x)^3} - \frac{17.5y}{(25-x)^2} + \frac{0.33541}{\sqrt{(25-x)y}} + \frac{0.65}{x-25}\right)
% \]

\[\quad \left(\frac{\partial z_{\text{top,new}}}{\partial y}\right)^2 = 0.0169 - \frac{0.12y^{\frac{5}{2}}}{c(x)^{\frac{5}{2}}} + \frac{0.252y^{\frac{3}{2}}}{c(x)^{\frac{3}{2}}} + \frac{0.16y^6}{c(x)^6} - \frac{0.672y^5}{c(x)^5} + \frac{1.2656y^4}{c(x)^4} - \frac{1.072y^3}{c(x)^3} + \frac{0.2716y^2}{c(x)^2}\]
\[+ \frac{0.182y}{c(x)} - \frac{0.21\sqrt{y}}{\sqrt{c(x)}} - \frac{0.039\sqrt{c(x)}}{\sqrt{y}} - \frac{0.0045x}{y} + \frac{0.1125}{y} \]

\[ L_{\text{top,new}} = \int_0^{5 - \frac{x}{5}} \sqrt{1 + \left( \frac{\partial z_{\text{top,new}}}{\partial y}\right)^2} \, \mathrm{d}y\]


Due to this integral being computationally intensive, an approximation is applied.

The integral is given by:
\[
I = \int_0^{5 - \frac{5}{x}} \sqrt{1 + f(y)} \, \mathrm{d}y,
\]
where
\[
f(y) = \left(\frac{\partial z_{\text{top,new}}}{\partial y}\right)^2
\]

Using the approximation of its Taylor series:
\[
\sqrt{1 + f(y)} \approx 1 + \frac{f(y)}{2},
\]
the integrand simplifies to:
\[
\int_0^{5 - \frac{5}{x}} \sqrt{1 + f(y)} \, dy \approx \int_0^{5 - \frac{5}{x}} \left(1 + \frac{f(y)}{2}\right) \, dy.
\]



To simplify \(f(y)\) while retaining accuracy, we select dominant terms based on their contributions to the integral. The constant term \(0.0169\) is retained because it contributes uniformly over the entire integration range. The term \(\frac{1.2656 y^4}{c(x)^4}\) is included because higher powers of \(y\) grow significantly as \(y \to 5 - \frac{5}{x}\), and its relatively large coefficient amplifies its contribution, though it is moderated by \(c(x)^4\). The linear term \(\frac{0.182 y}{c(x)}\) is chosen because it grows steadily with \(y\) and contributes over the full range. The inverse term \(\frac{0.1125}{y}\) is not concluded, as it diverges near \(y \to 0\). Higher-order terms such as \(\frac{0.16 y^6}{c(x)^6}\) and \(\frac{-0.672 y^5}{c(x)^5}\) are excluded because their contributions are heavily suppressed by large powers of \(c(x)\) and their smaller coefficients. Similarly, terms involving square roots, such as \(-\frac{0.21 \sqrt{y}}{\sqrt{c(x)}}\), are omitted because they grow slower than linear or polynomial terms and have smaller coefficients. By focusing on these dominant terms, we balance simplicity and accuracy for the approximation.



\[
 L_{\text{top,new}} = \int_0^{5 - \frac{5}{x}} \sqrt{1 + f(y)} \, dy \approx \int_0^{5 - \frac{5}{x}} \left(1 + \frac{1}{2}\left(0.0169 + \frac{1.2656 y^4}{c(x)^4} + \frac{0.182 y}{c(x)} \right)\right) \, dy
\]



\[ L_{\text{top,new}} \approx \int_0^{5 - \frac{5}{x}} 1.00845 + \frac{0.6328y^4}{c(x)^4} + \frac{0.091y}{c(x)}  \, dy \]



\[
L_{\text{top,new}} \approx 1.00845 \left(5 - \frac{5}{x}\right) + 0.12656\left( 5 - \frac{5}{x}\right) + 0.0455\left( 5 - \frac{5}{x}\right) 
\]
\[ L_{\text{top,new}} \approx 1.18051\left( 5 - \frac{5}{x}\right)\]

$|\mathbf{v}_{\text{top,new}}|$ therefore is approximately

\[ \frac{1.18051|\mathbf{v}_{\infty}|\left( 5 - \frac{5}{x}\right)}{5 - \frac{5}{x}} = 1.18051|\mathbf{v}_{\infty}| \]
\[|\mathbf{v}_{\text{top,new}}|^2 = |\mathbf{v}_{\text{bot,new}}|^2 = 1.3936|\mathbf{v}_{\infty}|^2 \]

\begin{figure}[H]
    \centering
    \includegraphics[scale=0.45]{new_velocity.png}
    \caption{Velocity vectors drawn and labeled on a cross-section of new wing}
\end{figure}

\subsection*{8.3 Expressions for the Pressure $P(x, y)$}
Let $ P_{\text{top}}$ and $P_{\text{bot}}$ be the pressure at each point on the top and bottom of the current design of the wing respectively. Using Bernoulli's equation, 

\[P_{\text{top}} = P_{\infty} + \frac{\rho}{2}(|\mathbf{v}_{\infty}|^2 - |\mathbf{v}_{\text{top}}|^2) = P_{\infty} + \frac{\rho}{2}\left(|\mathbf{v}_{\infty}|^2 - 2.19|\mathbf{v}_{\infty}|^2\right) = P_{\infty} - 0.595\rho |\mathbf{v}_{\infty}|^2 \]


\[P_{\text{bot}} = P_{\infty} + \frac{\rho}{2}(|\mathbf{v}_{\infty}|^2 - |\mathbf{v}_{\text{bot}}|^2) = P_{\infty} + \frac{\rho}{2}\left(|\mathbf{v}_{\infty}|^2 - 1.0816|\mathbf{v}_{\infty}|^2\right) = P_{\infty} - 0.0408 \rho |\mathbf{v}_{\infty}|^2 \]

Since $ P_{\text{top,new}}$ and $P_{\text{bot,new}}$ are equivalent under horizontal flight, and would produce no lift, the airfoil is angled upwards 9 degrees \cite{angle} relative to the horizontal for optimal lift. 

\subsection*{8.3 Expressions for the Pressure $P(x, y)$}
Let \( P_{\text{top}} \) and \( P_{\text{bot}} \) be the pressure at each point on the top and bottom of the current design of the wing respectively. Using Bernoulli's equation:

\[
P_{\text{top}} = P_{\infty} + \frac{\rho}{2}(|\mathbf{v}_{\infty}|^2 - |\mathbf{v}_{\text{top}}|^2) = P_{\infty} + \frac{\rho}{2}\left(|\mathbf{v}_{\infty}|^2 - 2.19 |\mathbf{v}_{\infty}|^2\right) = P_{\infty} - 0.595 \rho |\mathbf{v}_{\infty}|^2,
\]

\[
P_{\text{bot}} = P_{\infty} + \frac{\rho}{2}(|\mathbf{v}_{\infty}|^2 - |\mathbf{v}_{\text{bot}}|^2) = P_{\infty} + \frac{\rho}{2}\left(|\mathbf{v}_{\infty}|^2 - 1.0816 |\mathbf{v}_{\infty}|^2\right) = P_{\infty} - 0.0408 \rho |\mathbf{v}_{\infty}|^2.
\]

Since \( P_{\text{top,new}} \) and \( P_{\text{bot,new}} \) are equivalent under horizontal flight and would produce no lift, the airfoil is angled upwards by \( 9^\circ \) \cite{angle} relative to the horizontal for optimal lift. When the airfoil is tilted at an angle of attack \( \alpha = 9^\circ \), the symmetry of the velocity distribution is broken. The velocity on the upper surface increases, while the velocity on the lower surface decreases. Using the previously calculated velocity for horizontal conditions (\( |\mathbf{v}_{\text{top,new}}|^2 = |\mathbf{v}_{\text{bot,new}}|^2 = 1.3936 |\mathbf{v}_\infty|^2 \)), the adjusted velocities due to the angle of attack are:

\[
|\mathbf{v}_{\text{top,new}}|^2 = 1.3936 |\mathbf{v}_\infty|^2 + k \alpha |\mathbf{v}_\infty|^2,
\]
\[
|\mathbf{v}_{\text{bot,new}}|^2 = 1.3936 |\mathbf{v}_\infty|^2 - k \alpha |\mathbf{v}_\infty|^2,
\]

where:
\begin{itemize}
    \item \( k \) is the lift-curve slope (\( k = 2\pi \, \text{per radian} \, \cite{slope} \)),
    \item \( \alpha = 9^\circ = 0.157 \, \text{rad} \).
\end{itemize}

Since \( 2\pi \approx 6.283 \):
\[
|\mathbf{v}_{\text{top,new}}|^2 = 1.3936 |\mathbf{v}_\infty|^2 + 6.283 \cdot 0.157 |\mathbf{v}_\infty|^2 = 2.380 |\mathbf{v}_\infty|^2,
\]
\[
|\mathbf{v}_{\text{bot,new}}|^2 = 1.3936 |\mathbf{v}_\infty|^2 - 6.283 \cdot 0.157 |\mathbf{v}_\infty|^2 = 0.4072 |\mathbf{v}_\infty|^2.
\]

Using Bernoulli's equation, the pressures on the upper and lower surfaces are calculated as follows:


\[
P_{\text{top,new}} = P_\infty + \frac{\rho}{2} \left(|\mathbf{v}_\infty|^2 - |\mathbf{v}_{\text{top,new}}|^2 \right),
\]
\[
P_{\text{top,new}} = P_\infty + \frac{\rho}{2} \left(|\mathbf{v}_\infty|^2 - 2.380 |\mathbf{v}_\infty|^2 \right),
\]
\[
P_{\text{top,new}} = P_\infty - 0.69\rho |\mathbf{v}_\infty|^2 .
\]

\[
P_{\text{bot,new}} = P_\infty + \frac{\rho}{2} \left(|\mathbf{v}_\infty|^2 - |\mathbf{v}_{\text{bot,new}}|^2 \right),
\]
\[
P_{\text{bot,new}} = P_\infty + \frac{\rho}{2} \left(|\mathbf{v}_\infty|^2 - 0.4072 |\mathbf{v}_\infty|^2 \right),
\]
\[
P_{\text{bot,new}} = P_\infty + 0.2964 \rho|\mathbf{v}_\infty|^2 .
\]


\subsection*{8.4 Expressions for the normal vector}

Expressions for each normal vector can be calculated with the cross product 

\[ \frac{\partial r}{\partial x} \times \frac{\partial r}{\partial y} \]

where $r$ is the parametrization of the surface. Using previous calculations, 

\[
\mathbf{k}_\text{top} = \frac{\partial \mathbf{r}_{\text{top}}}{\partial x} \times \frac{\partial \mathbf{r}_{\text{top}}}{\partial y} =
\begin{vmatrix}
\mathbf{i} & \mathbf{j} & \mathbf{k} \\
1 & 0 & \frac{\partial z_{\text{top}}}{\partial x} \\
0 & 1 & \frac{\partial z_{\text{top}}}{\partial y}
\end{vmatrix} = \langle -\frac{\partial z_{\text{top}}}{\partial x}, -\frac{\partial z_{\text{top}}}{\partial y}, 1 \rangle
\] 

\[\mathbf{k}_\text{top} = \frac{\partial \mathbf{r}_{\text{top}}}{\partial x} \times \frac{\partial \mathbf{r}_{\text{top}}}{\partial y} = \langle -\frac{5y^2}{(25 - x)^2}, \frac{10y}{25 - x}, 1 \rangle\]

\[\mathbf{k}_\text{bot} = \frac{\partial \mathbf{r}_{\text{bot}}}{\partial x} \times \frac{\partial \mathbf{r}_{\text{bot}}}{\partial y} =
\begin{vmatrix}
\mathbf{i} & \mathbf{j} & \mathbf{k} \\
1 & 0 & \frac{\partial z_{\text{bot}}}{\partial x} \\
0 & 1 & \frac{\partial z_{\text{bot}}}{\partial y}
\end{vmatrix} = \langle -\frac{\partial z_{\text{bot}}}{\partial x}, -\frac{\partial z_{\text{bot}}}{\partial y}, 1 \rangle \]

\[\mathbf{k}_\text{bot} = \frac{\partial \mathbf{r}_{\text{bot}}}{\partial x} \times \frac{\partial \mathbf{r}_{\text{bot}}}{\partial y} = \langle -\frac{10y^2}{(50 - 2x)^2}, -\frac{x + 10y - 25}{50 - 2x}, 1 \rangle\]

For the new design, the partial derivatives are:




% As stated before, the surface area is

% \[
% S = \int \int \left\| \frac{\partial \mathbf{r}}{\partial x} \times \frac{\partial \mathbf{r}}{\partial y} \right\| \, \mathrm{d}x \, \mathrm{d}y,
% \]
% where \( \frac{\partial \mathbf{r}}{\partial x} \) and \( \frac{\partial \mathbf{r}}{\partial y} \) are the partial derivatives of the position vector \( \mathbf{r}(x, y) \) with respect to \(x\) and \(y\) that represent the parameterized surface; in this case it would simply be 2 times the surface area of $z_{\text{top,new}}(x, y)$, as the design is symmetrical.

% \section*{7.1 Surface Area of $z_{\text{top,new}}(x, y)$}

% $z_{\text{top,new}}(x, y)$ can be parametrized as 

% \[
% \mathbf{r}(x, y) = \langle x, y, z_{\text{top,new}}(x, y) \rangle
% \]


\[
\frac{\partial \mathbf{r}_{\text{top,new}}}{\partial x} = \langle 1, 0, \frac{\partial z_{\text{top,new}}}{\partial x} \rangle, \quad
\frac{\partial \mathbf{r}_{\text{top,new}}}{\partial y} = \langle 0, 1, \frac{\partial z_{\text{top}}}{\partial y} \rangle,
\]
where:
\small{
\[\frac{\partial z_{\text{top,new}}}{\partial x}= \frac{\partial c(x)}{\partial x}\left( \frac{0.3\sqrt{y}}{2\sqrt{c(x)}} + \frac{0.35y^2}{c(x)^2} - \frac{0.56y^3}{c(x)^3} + \frac{0.3y^4}{c(x)^4} \right)\]
\[
= \frac{-37.5y^4}{(x-25)^4} - \frac{14y^3}{(x-25)^3} - \frac{1.75y^2}{(x-25)^2} - \frac{0.067082\sqrt{y}}{\sqrt{25-x}} \]
}

\[
\quad \frac{\partial z_{\text{top,new}}}{\partial y} = \frac{0.3\sqrt{c(x)}}{2\sqrt{y}} - 0.13 - \frac{0.7y}{c(x)} + \frac{0.84y^2}{c(x)^2} - \frac{0.4y^3}{c(x)^3} \]

\[ = -0.13 + \frac{50y^3}{(x - 25)^3} + \frac{21y^2}{(x - 25)^2} + \frac{3.5y}{x - 25} + \frac{0.067082\sqrt{25 - x}}{\sqrt{y}}\]

The cross product is:
\[
\mathbf{k}_\text{top,new} = \frac{\partial \mathbf{r}_{\text{top,new}}}{\partial x} \times \frac{\partial \mathbf{r}_{\text{top,new}}}{\partial y} =
\begin{vmatrix}
\mathbf{i} & \mathbf{j} & \mathbf{k} \\
1 & 0 & \frac{\partial z}{\partial x} \\
0 & 1 & \frac{\partial z}{\partial y}
\end{vmatrix} = \langle -\frac{\partial z_{\text{top,new}}}{\partial x}, -\frac{\partial z_{\text{top,new}}}{\partial y}, 1 \rangle.
\]



\[
\mathbf{k}_\text{top,new} = \scalebox{0.95}{$
\Bigg \langle \frac{37.5y^4}{(x-25)^4} + \frac{14y^3}{(x-25)^3} + \frac{1.75y^2}{(x-25)^2} + \frac{0.067082\sqrt{y}}{\sqrt{25-x}}, 
0.13 - \frac{50y^3}{(x - 25)^3} - \frac{21y^2}{(x - 25)^2} - \frac{3.5y}{x - 25} - \frac{0.067082\sqrt{25 - x}}{\sqrt{y}}, 1 \Bigg \rangle
$}
\]

Due to symmetry, $\mathbf{k}_\text{bot,new} = -\mathbf{k}_\text{top,new}$






% The magnitude of the cross product is:
% \[
% \left\| \frac{\partial \mathbf{r}_{\text{top}}}{\partial x} \times \frac{\partial \mathbf{r}_{\text{top}}}{\partial y} \right\| =
% \sqrt{1 + \left( \frac{\partial z_{\text{top,new}}}{\partial x} \right)^2 + \left( \frac{\partial z_{\text{top,new}}}{\partial y} \right)^2}.
% \]

% The surface area of the surface is:
% \[
% S = \int_{0}^{15} \int_{0}^{5 - \frac{x}{5}} \sqrt{1 + \left( \frac{\partial z_{\text{top,new}}}{\partial x} \right)^2 + \left( \frac{\partial z_{\text{top,new}}}{\partial y} \right)^2} \, \mathrm{d}y \, \mathrm{d}x.
% \]
% \[S = \]
\subsection*{8.5 Expressions for the Force $F_\text{p}$}

The force per unit area on each wing is given by 

\[ \mathbf{F}_\text{p} = P\mathbf{n}\]. 

For the current design, 
\[ \mathbf{F}_\text{top} = -P_\text{top}\mathbf{n} = -\left(P_{\infty} - 0.595 \rho |\mathbf{v}_{\infty}|^2\right)  \Bigg\langle -\frac{5y^2}{(25 - x)^2}, \frac{10y}{25 - x}, 1 \Bigg\rangle  \]

\[ \mathbf{F}_\text{bot} = P_\text{bot}\mathbf{n} = \left(P_{\infty} - 0.0408 \rho |\mathbf{v}_{\infty}|^2\right)  \Bigg\langle -\frac{10y^2}{(50 - 2x)^2}, -\frac{x + 10y - 25}{50 - 2x}, 1 \Bigg\rangle  \]

For the new design, 

\[ \mathbf{F}_\text{top,new} = -P_\text{top,new}\mathbf{n} = -\left(P_\infty - 0.69 \rho |\mathbf{v}_\infty|^2\right) \mathbf{k}_\text{top,new}   \]

\[ \mathbf{F}_\text{bot, new} = P_\text{bot,new}\mathbf{n} = \left( P_\infty + 0.2964 \rho|\mathbf{v}_\infty|^2 \right)\mathbf{k}_\text{bot,new} \]

\subsection*{8.6 Calculations for Lift Generated}
Given that 
\[ P_\infty = 2116.8 \frac{\text{lb}}{\text{ft}^2}\]
\[ \rho = 0.07651 \frac{\text{lb}}{\text{ft}^3}\]
\[ \mathbf{v}_\infty = \langle 0, 146.67, 0 \rangle \frac{\text{ft}}{\text{sec}}\]

\[F_\text{top} = -(2116.8 - 0.595(0.07651)146.67^2)\Bigg\langle -\frac{5y^2}{(25 - x)^2}, \frac{10y}{25 - x}, 1 \Bigg\rangle\]

\[F_\text{top} = -1137.4955\Bigg\langle -\frac{5y^2}{(25 - x)^2}, \frac{10y}{25 - x}, 1 \Bigg\rangle\]

\[F_\text{bot} = (2116.8 - 0.0408(0.07651)146.67^2)\Bigg\langle -\frac{10y^2}{(50 - 2x)^2}, -\frac{x + 10y - 25}{50 - 2x}, 1 \Bigg\rangle \]

\[F_\text{bot} = 2049.6477\Bigg\langle -\frac{10y^2}{(50 - 2x)^2}, -\frac{x + 10y - 25}{50 - 2x}, 1 \Bigg\rangle \]

Therefore, 
\[F_\text{lift} = \int_0^{15} \int_0^{5 - \frac{x}{5}} (F_\text{top} \cdot \mathbf{k}) \mathrm{d}y \, \mathrm{d}x + \int_0^{15} \int_0^{5 - \frac{x}{5}} (F_\text{bot} \cdot \mathbf{k}) \mathrm{d}y \, \mathrm{d}x  \]

\[F_\text{lift} = -\int_0^{15} \int_0^{5 - \frac{x}{5}} 1137.4955 \, \mathrm{d}y \, \mathrm{d}x + \int_0^{15} \int_0^{5 - \frac{x}{5}} 2049.6477 \, \mathrm{d}y \, \mathrm{d}x\]

\[F_\text{lift} = -1137.4955\int_0^{15} 5 - \frac{x}{5} \, \mathrm{d}x + 2049.6477\int_0^{15} 5 - \frac{x}{5} \, \mathrm{d}x\]

\[F_\text{lift} = -1137.4955\left( 5(15) - \frac{15^2}{10}
\right)+ 2049.6477\left( 5(15) - \frac{15^2}{10}\right)\]

\[ F_\text{lift} = 47888 \, \text{lbs}\]

For the new design, 

\[ F_{\text{top,new}} = -(2116.8 - 0.69(0.07651)146.67^2)\mathbf{k_{top,new}}\]

\[F_{\text{top,new}} = -981.13 \, \mathbf{k_{top,new}} \]

\[ F_{\text{bot,new}} = (2116.8 + 0.2964(0.07651)146.67^2)\mathbf{k_{top,new}}  \]
\[ F_{\text{bot,new}} = 2604.6418 \, \mathbf{k_{top,new}}\]

\[ F_{\text{lift,new}} = \int_0^{15} \int_0^{5 - \frac{x}{5}} (F_\text{top,new} \cdot \mathbf{k}) \mathrm{d}y \, \mathrm{d}x + \int_0^{15} \int_0^{5 - \frac{x}{5}} (F_\text{bot,new} \cdot \mathbf{k}) \mathrm{d}y \, \mathrm{d}x  \] 

\[ F_{\text{lift,new}} = -\int_0^{15} \int_0^{5 - \frac{x}{5}} 981.13 \, \mathrm{d}y \mathrm{d}x + \int_0^{15} \int_0^{5 - \frac{x}{5}} 2604.6418 \,  \mathrm{d}y \mathrm{d}x \]

\[F_\text{lift} = -981.13\left( 5(15) - \frac{15^2}{10}
\right)+ 2604.6418\left( 5(15) - \frac{15^2}{10}\right)\]

\[  F_{\text{lift,new}} = 85234.4 \, \text{lbs} \]

The calculated lift far exceeds the manufacturing claim of 9000 lbs per wing, by a factor of 5.32. The lift of the new design exceeds the current design by a factor of 1.78. Now, consider an operating speed of 80 mph or 117.3 feet per second. At sea level, 

\[ P_\infty = 2116.8 \frac{\text{lb}}{\text{ft}^2}\]
\[ \rho = 0.07651 \frac{\text{lb}}{\text{ft}^3}\]
\[ \mathbf{v}_\infty = \langle 0, 117.3, 0 \rangle \frac{\text{ft}}{\text{sec}}\]

\[F_\text{top} = -(2116.8 - 0.595(0.07651)117.3^2)\Bigg\langle -\frac{5y^2}{(25 - x)^2}, \frac{10y}{25 - x}, 1 \Bigg\rangle\]

\[F_\text{top} = -1490.43\Bigg\langle -\frac{5y^2}{(25 - x)^2}, \frac{10y}{25 - x}, 1 \Bigg\rangle\]

\[F_\text{bot} = (2116.8 - 0.0408(0.07651)117.3^2)\Bigg\langle -\frac{10y^2}{(50 - 2x)^2}, -\frac{x + 10y - 25}{50 - 2x}, 1 \Bigg\rangle \]

\[F_\text{bot} = 2073.849\Bigg\langle -\frac{10y^2}{(50 - 2x)^2}, -\frac{x + 10y - 25}{50 - 2x}, 1 \Bigg\rangle \]

\[F_\text{lift} = -\int_0^{15} \int_0^{5 - \frac{x}{5}} 1490.43 \, \mathrm{d}y \, \mathrm{d}x + \int_0^{15} \int_0^{5 - \frac{x}{5}} 2073.849 \, \mathrm{d}y \, \mathrm{d}x\]

\[ F_\text{lift} = 30629 \, \text{lbs}\]

For the new design, 

\[ F_{\text{top,new}} = -(2116.8 - 0.69(0.07651)117.3^2)\mathbf{k_{top,new}}\]

\[F_{\text{top,new}} = -1390.42 \, \mathbf{k_{top,new}} \]

\[ F_{\text{bot,new}} = (2116.8 + 0.2964(0.07651)117.3^2)\mathbf{k_{top,new}}  \]
\[ F_{\text{bot,new}} = 2428.827 \, \mathbf{k_{top,new}}\]



\[ F_{\text{lift,new}} = -\int_0^{15} \int_0^{5 - \frac{x}{5}} 1390.42 \, \mathrm{d}y \mathrm{d}x + \int_0^{15} \int_0^{5 - \frac{x}{5}} 2428.827 \,  \mathrm{d}y \mathrm{d}x \]
\[  F_{\text{lift,new}} = 54516.4 \, \text{lbs} \] 
At this operating speed, the new design still outperforms the current design by the same scale factor. Based on these comparisons, I hypothesize that increased air speed, an angle of attack, and a thinner, more streamlined design would increase lift.

\section*{Conclusion}
The project revealed that redesigning the airfoil using a modified NACA 0015 profile significantly improved lift performance while adhering to material constraints and maintaining aerodynamic safety. The current design generated a lift of 47,888 lbs at 200 mph, while the new design achieved 85,234 lbs—an increase of 78 percent. At a reduced speed of 80 mph, the new design maintained the same improvement ratio, producing 54,516 lbs of lift compared to 30,629 lbs for the current design. This demonstrates that factors like airspeed, angle of attack, and streamlined geometry substantially influence lift generation. The results validate the effectiveness of the redesigned airfoil in achieving better aerodynamic efficiency and performance. In doing this project, I realized how integral mathematics is to the airline industry and its operations, being able to design efficient airfoils and wings and calculating aerodynamic forces to create models and projections . This project demonstrated that mathematics is not just a theoretical tool but a vital component in solving real-world challenges, driving innovation, and improving the efficiency and safety of modern aviation.











\bibliographystyle{unsrt}
\begin{thebibliography}{2}
\bibitem{airfoil_intro}
Wikipedia Contributors. “NACA Airfoil.” Wikipedia, Wikimedia Foundation, 28 Apr. 2019, 
\url{en.wikipedia.org/wiki/NACA_airfoil}
\bibitem{angle}
Kepekci, Haydar. (2022). Comparative Numerical Aerodynamics Performance Analysis of NACA0015 and NACA4415 Airfoils. International Journal of Engineering, Science and Information Technology. 2. 144-151. 10.52088/ijesty.v2i1.236.  
\url{https://www.researchgate.net/publication/357930405_Comparative_Numerical_Aerodynamics_Performance_Analysis_of_NACA0015_and_NACA4415_Airfoils}
\bibitem{slope}
Snorri Gudmundsson, Chapter 9 - The Anatomy of the Wing, Editor(s): Snorri Gudmundsson, General Aviation Aircraft Design, Butterworth-Heinemann,
2014, Pages 299-399, ISBN 9780123973085,
https://doi.org/10.1016/B978-0-12-397308-5.00009-X  
\url{https://www.sciencedirect.com/science/article/abs/pii/B978012397308500009X}




\end{thebibliography}

\end{document}


%sources: http://airfoiltools.com/airfoil/details?airfoil=naca2412-il

%https://en.wikipedia.org/wiki/NACA_airfoil

%https://www.researchgate.net/figure/cL-cD-values-for-various-angles-of-attack-for-NACA-0015_tbl2_357930405
